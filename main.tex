\documentclass[11pt]{article}
\usepackage{ctex}

\newcommand\xor{\oplus}
\newcommand\numberthis{\addtocounter{equation}{1}\tag{\theequation}}
\newcommand\p{\pause}
\newcommand\s{\section}
\newcommand\subs{\subsection}
\newcommand\be{\begin}
\newcommand\en{\end}
% \newcommand\todo{\ \\\textcolor{red}{\textbf{TODO:}} }


% Important!!!
\title{\huge \heiti 简易版题目}



\date{\today}
\author{张若天}

\be{document}

\maketitle



\leftline{编译开关}
% 开不开O2呢

开不开O2呢?


\thispagestyle{empty}



\newpage

\noindent{\emph{浮生有梦三千场\\
穷尽千里诗酒荒\\
徒把理想倾倒\\
不如早还乡\\
~\\
温一壶风尘的酒\\
独饮往事迢迢\\
举杯轻思量\\
泪如潮 青丝留他方\\
}}

\rightline{\emph{--- 乌糟兽/愚青《旧词》}}





\newpage


\s*{待定} %

\emph{tree.in/.out/.cpp}

\emph{时间限制:$\max$(1s, std两倍时间的上取整) / 空间限制:512MB}

\subs*{【问题描述】}

给定一棵$n$个点的有根树,节点标号$1\sim n$,$1$号节点为根。

给定$Q$个询问,每次询问给定$x,y$。

求$$\sum_{i \le x}{depth(lca(i,y))^k}$$

$lca(x,y)$表示节点$x$与节点$y$在有根树的最近公共祖先。

$depth(x)$表示$x$节点的深度,根节点的深度为$1$。

由于答案可能很大,所以请输出答案模$998244353$的结果。

\subs*{【输入格式】}

输入包含$n+Q$行。

第$1$行,三个正整数$n,Q,k$。

第$2 \sim n$ 行: 第i行有一个正整数$fa_i$,表示编号为$i$的节点的父亲节点编号。

接下来$Q$行,每行两个正整数$x,y$,表示一次询问。


\subs*{【输出格式】}

输出包含$Q$行,每行一个整数表示答案模$998244353$的结果。

\subs*{【样例输入】}

\begin{verbatim}
5 5 2
1
1
3
2
4 3
5 4
2 5
1 2
3 2
\end{verbatim}

\subs*{【样例输出】}

\begin{verbatim}
10
16
5
1
6
\end{verbatim}


\subs*{【样例解释】}

输入的树:
\begin{verbatim}
1
|\
2 3 - 4
|
5
\end{verbatim}

每个点的$depth$分别为 $1, 2, 2, 3, 3$。

第一个询问$x=4, y=3$:
$lca(1,3) = 1$, $lca(2,3) = 1$, $lca(3,3) = 3$, $lca(4,3) = 3$;
$depth(1)^2 + depth(1)^2 + depth(3)^2 + depth(3)^2 = 1+1+4+4 = 10$。

所以输出为10。

\subs*{【数据规模和约定】}


对于20\%的数据,满足$n,Q \le 2000$。

对于另外20\%的数据,存在某个点的深度为$n$;

对于另外10\%的数据,$Q=n$,且对于第i个询问,$x=i$;

对于另外10\%的数据,$k=1$;

对于另外10\%的数据,$k=2$;

对于另外10\%的数据,$k=3$;

对于100\%的数据,满足$1 \le n,Q \le 50000 ,  1 \le k \le 10^9$ , $1 \le x,y,fa_i \le n$。

\newpage





(完)



\en{document}



\s*{Prob} %

\emph{.in/.out/.cpp}

\subs*{【问题描述】}


\subs*{【输入格式】}

\subs*{【输出格式】}

\subs*{【样例输入】}

\begin{verbatim}

\end{verbatim}

\subs*{【样例输出】}

\begin{verbatim}

\end{verbatim}


\subs*{【数据规模和约定】}

对于30\%的数据,满足$n \le 15$;

对于60\%的数据,满足$n \le 1000$;

对于100\%的数据,满足$n \le 100000 ,  1\le a_i \le 10^9$。

\newpage


