\documentclass[11pt]{article}
\usepackage{ctex}

\newcommand\xor{\oplus}
\newcommand\numberthis{\addtocounter{equation}{1}\tag{\theequation}}
\newcommand\p{\pause}
\newcommand\s{\section}
\newcommand\subs{\subsection}
\newcommand\be{\begin}
\newcommand\en{\end}
% \newcommand\todo{\ \\\textcolor{red}{\textbf{TODO:}} }


% Important!!!
\title{\huge \heiti 简易版题解}



\date{\today}
\author{张若天}

\be{document}

\maketitle

$$\sum_{i \le x}{depth(lca(i,y))^k}$$

\s{首先考虑$k=1$}

求的是$\sum_{i \le x}{depth(lca(i,y))}$,一堆点然后每个点和y求lca然后深度求和。

总体思路是把lca的值摊派到这个点到根的路径上(这个东西也叫树上差分?),再离线解决所有询问。

维护一个点权sum[],初始为0,然后将y到根的每个点的点权设为1,然后对于每个点$i\le x$,求从i到根的权值和为上面要求的答案,但这样就$O(n)$了。

(可以反向考虑),维护一个点权sum[],初始为0,对于小于等于x的点i,将i到根的路径上所有点的点权++。然后求从y到根的权值和也是上面要求的答案。这种方法求可以按x排序,然后离线,x相等的询问一块问。

可以树链剖分+线段树解决。$O(nlog^2 n)$。或者LCT也行。

\s{然后考虑$k>1$}

k=2的话,按照上述思路想,把$lca^2$的值摊到到根的路径上的话就不是之前的1,1,1,...,变成了1,3,5,7... 
直接看的话问题变成了线段树区间加等差数列,好像改一下线段树实现也能做(所以给了点部分分)。

但是k>2的时候就比较麻烦了。

基于把$lca^k$摊到从这个点到根的路径上这个思路,实际上对于深度是x的点来说,这个点每次点权增加的值固定是$x^k - (x-1)^k$。

所以实际上,线段树打标记只用记录每个点被算了多少次cnt[]即可。然后实际上的权值和是$sum[i]=cnt[i]*(dep[i]^k - (dep[i]-1)^k)$,每次操作只有cnt[]区间加1,于是预处理线段树上每个区间的$\sum{(dep[i]^k - (dep[i]-1)^k)}$后就可以直接拿线段树维护sum[]。

于是还是之前的树链剖分+线段树解决。$O(nlog^2 n)$。或者LCT也行。


\newpage





(完)



\en{document}



\s*{Prob} %

\emph{.in/.out/.cpp}

\subs*{【问题描述】}


\subs*{【输入格式】}

\subs*{【输出格式】}

\subs*{【样例输入】}

\begin{verbatim}

\end{verbatim}

\subs*{【样例输出】}

\begin{verbatim}

\end{verbatim}


\subs*{【数据规模和约定】}

对于30\%的数据,满足$n \le 15$;

对于60\%的数据,满足$n \le 1000$;

对于100\%的数据,满足$n \le 100000 ,  1\le a_i \le 10^9$。

\newpage


